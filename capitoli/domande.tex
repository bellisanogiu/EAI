\chapter{Le domande da esame}

\section{Immagini digitali}
\begin{itemize}
	\item Image processing, analysis e understanding. Spiegarne la differenza
	\item Definire una immagine digitale
	\item Cosa si intende per risoluzione di una immagine?
	\item Illustrare il processo di campionamento e quantizzazione
	\item Descrivere i passi fondamentali di un processo di elaborazione di immagini digitali
\end{itemize}

\section{Concetti base sulle immagini digitali}
\begin{itemize}
	\item Come si classificano le operazioni spaziali da applicare su un’immagine digitale?
	\item Definire il concetto di adiacenza nelle immagini digitali
	\item Definire una regione di un’immagine binaria
	\item Definire un contorno in un’immagine binaria
	\item Definire un edge in un’immagine digitale
	\item Illustrare il concetto di distanza tra punti in una immagine binaria
	\item Fornire la definizione di istogramma di un’immagine
	\item Illustrare le strutture dati tradizionali per rappresentare un’immagine
	\item Illustrare le strutture dati topologiche per rappresentare un’immagine
	\item Illustrare le strutture dati gerarchiche per rappresentare un’immagine
\end{itemize}

\section{Pre-processing, filtri spaziali, trasformazioni}
\begin{itemize}
	\item Cosa si intende per preprocessing di una immagine
	\item Descrivere i metodi di preprocessing di una immagine
	\item Descrivere le trasformazioni geometriche
	\item Cosa si intende per filtro spaziale?
	\item Cosa si intende per trasformata di una immagine?
	\item Cosa si intende per equalizzazione di un istogramma? A cosa serve?
	\item Come si può realizzare uno stretching del contrasto presente in una immagine?
	\item Definire un filtro spaziale lineare
	\item Cosa si intende per operatore lineare?
	\item Correlazione e convoluzione spaziale. Spiegarne le differenze
	\item Descrivere i filtri per lo smoothing nel dominio spaziale
\end{itemize}

\section{Edge detection}
\begin{itemize}
	\item Illustrare i differenti tipi di edge presenti in un’immagine
	\item Classificazione degli edge detector
	\item Illustrare i metodi per l’edge detection basati sull’approssimazione delle derivate prime.
	\item Descrivere la tecnica della non-maxima suppression
	\item Illustrare il Marr-Hildreth edge detector
	\item Illustrare il Canny edge detector
	\item Quali sono i vantaggi e gli svantaggi degli edge detector?
	\item Quali sono i criteri per valutare la bontà di un edge detector?
	\item Descrivere le tecniche locali e regionali per l’edge linking
	\item Illustrare la trasformata di Hough
\end{itemize}

\section{Segmentazione}
\begin{itemize}
	\item Definizione di segmentazione. Metodi e problemi
	\item Illustrare la segmentazione basata su thresholding
	\item Descrivere il metodo iterativo per la segmentazione mediante soglia globale
	\item Descrivere il metodo per la segmentazione mediante soglia ottimale
	\item Illustrare il metodo region growing per la segmentazione
	\item Illustrare il metodo region splitting and merging per la segmentazione
	\item Descrivere la segmentazione mediante watershed morfologico
\end{itemize}

\section{Descrizione e rappresentazione}
\begin{itemize}
	\item Cosa si intende per rappresentazione e descrizione di un oggetto?
	\item Descrivere i metodi per la rappresentazione del contorno
	\item Illustrare i descrittori del contorno
	\item Illustrare i descrittori semplici di una regione
	\item Illustrare i descrittori topologici di una regione
	\item Illustrare i descrittori basati sulla texture
\end{itemize}

\section{Trasformata di Fourier}
\begin{itemize}
	\item Illustrare la rappresentazione di un’immagine nel dominio spaziale e nel dominio delle frequenze
	\item Fornire la definizione di funzione armonica
	\item Descrivere brevemente la trasformata di Fourier
	\item Quali sono le proprietà della DFT?
	\item Spettro di Fourier, angolo di fase e spettro di potenza. Fornire la loro definizione
	\item Illustrare il teorema di convoluzione
	\item Quali sono i passi da eseguire per il filtraggio nel dominio delle frequenze?
	\item Illustrare la corrispondenza tra filtraggio nel dominio spaziale e filtraggio nel dominio delle frequenze
	\item Descrivere i filtri per lo smoothing nel dominio delle frequenze
	\item Descrivere i filtri per lo sharpening nel dominio delle frequenze
	\item Descrivere i filtri selettivi nel dominio delle frequenze
\end{itemize}

\section{Rumore}
\begin{itemize}
	\item Descrivere il rumore e le sue proprietà
	\item Illustrare i filtri per il rumore (basati sulla media)
	\item Illustrare i ranking filter
	\item Riduzione del rumore periodico mediante filtraggio nel dominio delle frequenze
\end{itemize}

\section{Morfologia matematica}
\begin{itemize}
	\item Cosa è la Morfologia matematica?
	\item In cosa consiste una operazione morfologica?
	\item Fornire la definizione di elemento strutturante nella morfologia matematica
	\item Definire l’erosione e la dilatazione nella morfologia binaria. Illustrarne le proprietà
	\item Definire l’opening e il closing nella morfologia binaria. Illustrarne le proprietà
	\item Definire la trasformazione Hit-or-Miss della morfologia binaria
	\item Come si può realizzare un filtro morfologico?
	\item Illustrare l’algoritmo per l’estrazione del contorno mediante operatori morfologici
	\item Illustrare l’algoritmo per il riempimento di buchi (hole filling) mediante operatori morfologici
	\item Illustrare l’algoritmo per l’estrazione delle componenti connesse mediante operatori morfologici
	\item Illustrare l’algoritmo per l’individuazione della convex hull mediante operatori morfologici
	\item Illustrare l’algoritmo per il thinning mediante operatori morfologici
	\item Illustrare l’algoritmo per il thickening mediante operatori morfologici
	\item Illustrare l’algoritmo per l’estrazione dello skeleton mediante operatori morfologici
	\item Illustrare l’algoritmo per il pruning mediante operatori morfologici
	\item Definire l’erosione e la dilatazione geodesica nella morfologia binaria
	\item Illustrare la ricostruzione morfologica
	\item Illustrare alcune applicazioni della ricostruzione morfologica binaria
	\item Definire l’erosione e la dilatazione nella morfologica in toni di grigio
	\item Definire l’opening e il closing nella morfologia in toni di grigio
	\item Illustrare lo smoothing morfologico
	\item Definire il gradiente morfologico
	\item Descrivere le trasformazioni morfologiche top-hat e bottom-hat
	\item Descrivere la granulometria morfologica
	\item Illustrare la ricostruzione morfologica in tono di grigio
\end{itemize}

\section{Image Processing Toolbox and Matlab}
\begin{itemize}
	\item Quali sono le Data classes per rappresentare un’immagine in Matlab
	\item Quali sono i tipi di immagini supportate da Matlab
	\item Lettura, visualizzazione e salvataggio di immagini in Matlab
	\item Metodi per il miglioramento del contrasto in Matlab
	\item Filtri lineari in Matlab. Come si implementano?
	\item Filtri non lineari in Matlab. Come si implementano?
	\item Edge detection in Matlab. Come si implementa?
	\item Rumore in Matlab. Come si gestisce?
	\item Come si implementa in Matlab la segmentazione mediante soglia?
	\item Descrivere il metodo per il calcolo della DFT 2D in Matlab
	\item Descrivere il metodo per la visualizzazione della DFT 2D in Matlab
	\item Come si crea un elemento strutturante in Matlab?
	\item Descrivere i metodi per realizzare gli operatori morfologici in Matlab
	\item Come si etichettano le componenti connesse in Matlab?
	\item A cosa serve la funzione Matlab bwmorph?
	\item A cosa serve la funzione Matlab imreconstruct?
\end{itemize}