\chapter{Segmentazione}

\section{Formulazione matematiche di base}

\begin{itemize}
	\item \textit{R} rappresenta l'intera regione dell'immagine
	\item la segmentazione è il processo che divide la regione \textit{R} nelle sottoregioni $R_1, R_2, \dots, R_n$ per cui valgono
	\begin{enumerate}
		\item $\bigcup_{i=1}^{n} R_i = R$ \quad ovvero la somma di tutte le sottoregioni è uguale all'intera regione \textit{R} dell'immagine
		
		\item $R_i$ è un insieme \gls{connesso} dove $i = 1, 2, \dots, n$
		\item $R_i \cap R_j = \emptyset \quad \forall i,j \quad i \neq j$
		\item $Q(R_i) = TRUE \quad i = 1, 2, \dots, n$
		\item $Q(R_i \cup R_j) = FALSE \quad \text{per ogni regione adiacente } R_i$ e $R_j$
	\end{enumerate}
\end{itemize}

\section{Region growing}
E' una procedura che raggruppa i pixel o le sottoregioni in regioni più grandi basandosi su criteri predefiniti.

L'approccio base è il seguente:

\begin{itemize}
	\item si parte con un insieme di punti "seme"
	\item da queste grow regioni si aggiungono ad ogni seme i suoi pixel vicini che hanno dele predefinite proprietà simili a quelle del seme (come uno specifico range di intensità o un colore)
	\item è importante selezionare con attenzione i punti \textit{da cui partire} in base al tipo di problema o all'immagine da processare.
	\begin{itemize}
			\item le immagini satellitari dipendono dal colore
			\item le immagini monocromatiche dipendono dai descrittori basati sui livelli di intensità e sulle proprietà spaziali (texture, momenti)
	\end{itemize}
	
	\item devono essere usate le proprietà della connessione
	\item si deve formulare una regola di stop
\end{itemize}

\section{Qual è l'algoritmo di un region growing?}

\begin{itemize}
	\item si assume che:

\begin{itemize}
	\item $f(x, y)$ è un'immagine in ingresso
	\item $S(x, y)$ indica un array di semi contenente un 1s nei posti dei punti del seme e 0 negli altri punti
	\item $Q$ indica un predicato che verrà applicato ad ogni locazione $(x, y)$
\end{itemize}

	\item si assume che gli array $f$ ed $S$ abbiano la stessa dimensione
	\item si utilizza una 8-connessione
\end{itemize}

Con queste premesse si applica il seguente algoritmo:

\begin{enumerate}
	\item si trovano tutte le componenti in $S(x, y)$
	
	\begin{itemize}
		\item si erode ogni componente connessa sino ad ottenere un solo pixel
		\item si etichetta (label) ogni pixel trovato con il valore $1$
		\item si etichettano tutti gli altri pixel in $S$ con il valore $0$
	\end{itemize}
	
	\item si realizza un'immagine $f_q$, tale che per ogni coppia di coordinate $(x, y)$ si abbia che:
	
		\begin{itemize}
			\item $F_q(x, y) = 1$ se l'immagine in ingresso soddisfa il predicato $Q$ a queste coordinate
			\item $f_q(x, y) = 0$ negli altri casi
		\end{itemize}
		
		\item si forma così un'immagine $g$ realizzata aggiungendo ad ogni punto seme in $S$ tutti i punti 1-valutati in $f_q$ che sono 8-connessioni al punto seme
		
		\item si etichetta ogni componente connessa in $g$ con una etichetta di regione differente (1, 2, 3, \dots). Questa è così l'immagine segmentata ottenuta attraverso l'algoritmo region growing
\end{enumerate}

\section{Region Splitting and Merging}
\subsection{Definizione}
Si tratta di una procedura che suddivide un'immagine prima di tutto in un insieme arbitrario di regioni disgiunte e quindi le fonde (merge) e/o le suddivide in regioni che soddisfino i requisiti della segmentazione.

\subsection{Region Splitting and Merging: algoritmo}
\begin{itemize}
	\item si assuma che:
	\begin{itemize}
		\item $R$ è la regione che rappresenta l'intera immagine
		\item sia $Q$ il predicato utilizzato
	\end{itemize}
	
	\item si parte con l'intera regione $R$ suddividendo questa in quadranti sempre più piccoli $R_i$, dove per ognuno di essi vale che $Q(R_i) = TRUE$. 
	\begin{itemize}
		\item se $Q(R) = FALSE$ si divide l'immagine in nuovi quadranti
		\item se $Q = FALSE$ per ogni quadrante, si suddivide il quadrante in ulteriori sotto-quadranti e così via
	\end{itemize}
	
\end{itemize}

\subsection{Region Splitting and Merging: tecniche di rappresentazione}

Una delle tecniche più utilizzata è denominata \textbf{Quadtree}:

\begin{itemize}
	\item si tratta di un albero in cui cui ogni nodo ha 4 discendenti
	\item la radice corrisponde all'intera immagine
	\item ogni nodo corrisponde a 4 foglie
\end{itemize}

\subsection{Caratteristiche}
\begin{itemize}
	\item se viene usato solo lo \textit{splitting} la partizione finale contiene regioni adiacenti con proprietà identiche
	
	\item soddisfare i vincoli di segmentazione richiede la	fusione solo delle regioni adiacenti i cui pixel soddisfano il predicato $Q$:
	
	\begin{itemize}
		\item due regioni adiacenti $R_i$ e $R_j$ sono fuse solo se vale: $Q(R_i \cup R_j) = TRUE$
	\end{itemize}
\end{itemize}

\subsection{Algoritmo per lo splitting and merging}
\begin{enumerate}
	\item si suddivide in 4 quadranti disgiunti ogni regione $R_j$ per cui $Q(R_i) = FALSE$
	
	\item quando non è possibile un ulteriore suddivisione, si fonde ogni regione $R_i$ e $R_j$ adiacente per cui $Q(R_i \cup R_j) = TRUE$
	
	\item ci si ferma quando non sono possibili ulteriori fusioni
\end{enumerate}

Si è soliti indicare una dimensione minima quadregion oltre il quale nessuna ulteriore frazionamento sia effettuato.

\textbf{una variante}: la fusione di due qualsiasi regioni adiacenti $R_i$ e $R_j$ se
ognuno soddisfa il predicato singolarmente.

\subsection{Region growing Matlab}

\begin{lstlisting}
function [g, NR, SI, TI] = regiongrow(f, S, T)
f = double(f);
% if S is a scalar, obtain the seed image
if numel(S) == 1
SI = f == S;
S1 = S;
else
SI = bwmorph(S, 'shrink', Inf);
J = find(SI);
S1 = f(J); % Array of seed values
end
TI = false(size(f));
for K = 1:length(S1)
seedvalue = S1(K);
S = abs(f - seedvalue) <= T;
TI = TI | S;
end
[g, NR] = bwlabel(imreconstruct(SI, TI));
\end{lstlisting}

S can be an array (the same size as f) with a 1
at the coordinates of every seed point and 0s
elsewhere. S can also be a single seed value.
(Our example S = 255)

Similarly, T can be an array (the same size as
f) containing a threshold value for each pixel in
f. T can also be a scalar, in which case it
becomes a global threshold.
(Our example T = 65)

g is the result of region growing, with each
region labelled by a different integer.

NR is the number of regions.

SI is the final seed image used by the
algorithm.

TI is the image consisting of the pixels in f
satisfied the threshold test.

Use function imreconstruct with SI as the
marker image to obtain the regions
corresponding to each seed in S.

bwlabel assigns a different integer to each
connected region.

\subsection{Region growing Matlab: le funzioni}
\begin{itemize}
\item la funzione predefinita per implementare il quadtree è \textit{qtdecomp} con la sintassi:

$$S = qtdecomp(f, @split_test, parameters)$$

\begin{itemize} 
\item $f$ è l'immagine di ingresso
\item $S$ è una matrice sparsa contenente la struttura quadtree

	\begin{itemize}
		\item se $S(k, m) \neq 0 $, allora $(K, m)$ è l'estremo superiore sinistro del blocco in decomposizione e la dimensione del blocco è data da $S(k, m)$
	\end{itemize}

\item la funzione $split_test$ è usata per determinare se una regione deve essere divisa o meno
\item $parameters$ indica eventuali parametri addizionali 
\end{itemize}
	
\item per ottenere i valori dei pixel della quadregion in una decomposizione quadtree, si utilizza la funzione \textbf{qtgetblk} con la sintassi:

$$[vals, r, c] = qtgetblk(f, S, m)$$

\begin{itemize}
	\item $vals$ è un array che contiene i valori dei blocchi di dimensione $mXm$ nella decomposizione quadtree $f$
	\item $S$ è la matrice sparsa restituita dalla funzione $qtdecomp$
	\item i parametri $r, c$ sono dei vettori che contengono le coordinate della riga e della colonna degli angoli in alto a sinistra dei blocchi
\end{itemize}

\item la funzione che implementa l'algoritmo di segmentazione è \textbf{splitmerge} che la seguente sintassi:

$$g = splitmerge(f, mindim, @predicate)$$

begin{itemize} 
\item $f$ è l'immagine di ingresso
\item $g$ è l'immagine in uscita
\item $mindim$ definisce la dimensione del più piccolo blocco nella decomposizione (potenza di 2)
\item $predicate$ è una funzione definita dall'utente che deve essere inclusa nell'ambiente (path) di Matlab. La sua sintassi è:
\begin{itemize}
	\item $flag = predicate(region)$
	\item restituisce $true$ (1) se i pixel nella regione soddisfa il predicato definito dal codice nella funzione
	\item restituisce $false$ (0) negli altri casi
	
	\item 
	\begin{lstlisting}
		function flag = predicate(region)
		sd = std2(region);
		m = mean2(region);
		flag = (sd > 10) & (m > 0) & (m < 125);
	\end{lstlisting}
\end{itemize}
\end{itemize}

\subsection{Region Splittin and Merging in Matlab: splitmerge}
\begin{lstlisting}
function g = splitmerge(f, mindim, fun)
Q = 2^nextpow2(max(size(f)));
[M, N] = size(f);
f = padarray(f, [Q - M, Q - N], 'post');

S = qtdecomp(f, @split_test, mindim, fun);

Lmax = full(max(S(:)));

g = zeros(size(f));
MARKER = zeros(size(f));

for K = 1: Lmax
	[vals, r, c] = qtgetblk(f, S, K);
	if ~isempty(vals)
		for I = 1:length(r)
			xlow = r(I);
			ylow = c(I);
			xhigh = xlow + K -1;
			yhigh = ylow + K -1;
			region = f(xlow:xhigh, ylow:yhigh);
			flag = feval(fun, region);
			if flag
				g(xlow:xhigh, ylow:yhigh) = 1;
				MARKER(xlow, ylow) = 1;
			end
		end
	end
end

g = bwlabel(imreconstruct(MARKER, g));
g = g(1:M, 1:N);
\end{lstlisting}

\subsection{Region Splitting and Merging in Matlab: split test}
\begin{lstlisting}
function v = split_test(B, mindim, fun)

k = size(B,3);
v(1:k) = false;

for I = 1:k
	quadregion = B(:, :, I);
	if size(quadregion, 1) <= mindim
		v(I) = false;
		continue;
	end
	
	flag = feval(fun, quadregion);
	if flag
		v(I) = true;
	end
end
\end{lstlisting}
