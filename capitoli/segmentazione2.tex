\chapter{Segmentazione}

\section{Formulazione matematiche di base}

\begin{itemize}
	\item \textit{R} rappresenta l'intera regione dell'immagine
	\item la segmentazione è il processo che divide la regione \textit{R} nelle sottoregioni $R_1, R_2, \dots, R_n$ per cui valgono
	\begin{enumerate}
		\item $\bigcup_{i=1}^{n} R_i = R$ \quad ovvero la somma di tutte le sottoregioni è uguale all'intera regione \textit{R} dell'immagine
		
		\item $R_i$ è un insieme \gls{connesso} dove $i = 1, 2, \dots, n$
		\item $R_i \cap R_j = \emptyset \quad \forall i,j \quad i \neq j$
		\item $Q(R_i) = TRUE \quad i = 1, 2, \dots, n$
		\item $Q(R_i \cup R_j) = FALSE \quad \text{per ogni regione adiacente } R_i$ e $R_j$
	\end{enumerate}
\end{itemize}

\section{Region growing}
E' una procedura che raggruppa i pixel o le sottoregioni in regioni più grandi basandosi su criteri predefiniti.

L'approccio base è il seguente:

\begin{itemize}
	\item si parte con un insieme di punti "seme"
	\item 
\end{itemize}