\chapter{Morfologia matematica scala di grigi}
Nella morfologia scala di grigi si utilizzano funzioni discrete definite in $Z^2$ nella forma:
\begin{itemize}
	\item $f(x, y)$ che indica una immagine in scala di grigi
	
	\item $b(x, y)$ che indica un SE, utilizzata come sonda per esaminare un'immagine e le sue specifiche proprietà
	\begin{itemize}
		\item gli SE nella morfologia scala di grigi può essere $flat$ oppure $nonflat$
		\item l'origine deve essere definita e solitamente risulta essere simmetrica con l'origine situata al centro
		
		\item la riflessione di SE è indicata come:
		$$
		\hat{b} (x, y) = b (-x, -y)
		$$
	\end{itemize}
\end{itemize}

\section{Erosione e dilatazione tramite un SE flat}
\begin{itemize}
	\item L'\textbf{erosione} di $f$ tramite un SE flat $b$ in ogni punto $(x, y)$ è definito come il \textbf{minimo} valore di uan immagine in una regione coincidente con $b$ quando l'origine di $b$ si trova in $(x, y)$:
	$$
	[f \ominus b] (x, y) = \displaystyle\min_{(s, t) \in b} \{f(x + s, y + t)\}
	$$
	
	\item la \textbf{dilazione} di $f$ tramite un SE flat $b$ in ogni punto $(x, y)$ è definito come il \textbf{massimo} valore di una immagine nella finestra delineata da $\hat{b}$ quando l'origine di $\hat{b}$ si trova in $(x, y)$:
	$$
	[f \oplus b] (x, y) = \displaystyle\max_{(s, t) \in b} \{f(x - s, y - t)\}
	$$
\end{itemize}

\section{Erosione e dilatazione tramite un SE nonflat}
\begin{itemize}
	\item l'\textbf{erosione} di $f$ tramite un SE nonflat $b_N$ è definito come:
	$$
	[f \ominus b_N] (x, y) = \displaystyle\min_{(s, t) \in b_N} \{f(x + s, y + t) - b_N(s, t)\}
	$$
	
	\item la \textbf{dilatazione} di $f$ tramite un SE nonflat $b_N$ è definito come:
	$$
	[f \oplus b_N] (x, y) = \displaystyle\max_{(s, t) \in b_N} \{f(x - s, y - t) + b_N(s, t)\}
	$$
	
	\item quando tutti gli elementi di $b_N$ sono costanti (per esempio l'SE è flat) la definizione si riduce alla precedente all'interno di una costante scalare uguale all'ampiezza di SE
	
	\item gli SE scala di grigi sono raramente utilizzati
	
	\item l'erosione e la dilatazione sono duali rispetto alla funzione di complemento e riflessione
\end{itemize}

\section{Opening e Closing}
\begin{itemize}
	\item l'opening è usato per rimuovere piccoli dettagli luminosi, mentre i dettagli più grandi vengono trascurati.
	
	\item l'opening di una immagine $f$ attraverso un SE $b$ è denotato come $f \circ b$:
	$$
	f \circ b = (f \ominus b) \oplus b
	$$
\end{itemize}

\begin{itemize}
	\item il closing è utilizzato per attenuare le caratteristiche scure con una attenuazione graduale che dipende dalle dimensioni delle caratteristiche rispetto a SE:
	
	\item il closing di una immagine $f$ tramite un SE $b$ è indicato come $f \bullet b = (f \oplus b) \ominus b$
\end{itemize}

L'opening e il closing sono duali rispetto al complemento e alla SE riflessione:
$$
(f \circ b)^c = f^c \bullet \hat{b} \quad \text{e} \quad (f \bullet b)^c = f^c \circ \hat{b}
$$

\subsection{Opening e Closing: proprietà}
\begin{itemize}
	\item l'opening scala di grigi soddisfa le seguenti proprietà:
	\begin{itemize}
		\item $f \circ b \rotatebox[origin=c]{180}{$\Rsh$} f$
		
		\item se $f_1 \rotatebox[origin=c]{180}{$\Rsh$}	f_2 \quad \text{allora} \quad (f_1 \circ b) \rotatebox[origin=c]{180}{$\Rsh$}(f_2 \circ b)$
		
		\item $(f \circ b) \circ b = f \circ b$ 
	\end{itemize}
	
	\item il closing scala di grigi soddisfa le seguenti proprietà:
	\begin{itemize}
		\item $f \rotatebox[origin=c]{180}{$\Rsh$} f \bullet b$
		
		\item se $f_1 \rotatebox[origin=c]{180}{$\Rsh$}	f_2 \quad \text{allora} \quad (f_1 \bullet b) \rotatebox[origin=c]{180}{$\Rsh$}	 (f_2 \bullet b)$
		
		\item $(f \bullet b) \bullet b = f \bullet b$
	\end{itemize}
\end{itemize}
NOTA: la notazione $e \rotatebox[origin=c]{180}{$\Rsh$}	 r$ è usata per indicare che il dominio di $e$ è un sottoinsieme del dominio di $r$ e che $e(x, y) \leq r(x,y)$ per ogni $(x, y)$ ne dominio di $e$

\section{Algoritmi basati sulla scala di grigi morfologica}

\subsection{Smoothing morfologico}
L'opening elimina piccoli dettagli luminosi, piccoli rispetto ad uno specifico SE. Il closing invece elimina i dettagli scuri. Essi sono usati in combinazione come filtri morfologici per lo smoothing dell'immagine e per rimuovere il rispettivo rumore.

In maniera alternativa, è possibile ottenere una sequenza di filtering nel seguente modo:
La sequenza di opening-closing incomincia con l'immagine originale, ma i singoli passi dell'opening e del closing vengono effettuati sul risultato del passo precedente.

\subsection{Gradiente morfologico}
La dilatazione e l'erosione possono essere usate in combinazione con la "sottrazione dell'immagine" per ottenere il gradiente morfologico $g$ di una immagine:
$$
g = (f \oplus b) - (f \ominus b)
$$

La dilatazione addensa le regioni in una immagine e l'erosione le restringe.

La loro differenza enfatizza i confini tra le regioni.

\subsection{Trasformazioni Top-hat e Bottom-hat}
\begin{itemize}
	\item la trasformazione top-hat di una immagine in toni di grigio $f$ è definita come "f minus its opening":
	$$
	T_hat(f) = f - (f \circ b)
	$$
	
	\item la trasformazione bottom-hat di una immagine in toni di grigio $f$ è definita come il closing di $f$ minus $f$:
	$$
	B_hat(f) = (f \bullet b) - f
	$$
\end{itemize}

Queste trasformazioni sono usate per rimuovere gli oggetti dall'immagine usando un SE nelle operazioni di opening o closing che non si adatta alle immagini da rimuovere.

L'operazione differenza produce un'immagine in cui rimangono solo i componenti rimossi:
\begin{itemize}
	\item la trasformazione top-hat è usata per gli oggetti luminosi su uno sfondo scuro
	\item la trasformazione bottom-hat è usata per il contrario.
\end{itemize}

\subsection{Granulometria}
E' un campo che si occupa di determinare la dimensione della distribuzione delle particelle in una immagine.

La morfologia può essere usata per stimare la dimensione della distribuzione di particelle nella distribuzione senza identificare e misurare ogni particella nell'immagine.

Con particelle aventi forma regolare e che sono più luminose rispetto allo sfondo, il metodo consiste nell'applicare l'opening con SE di dimensioni maggiori.

Le operazioni di opening con una certa dimensione hanno un effetto maggiore nelle regioni dell'immagine di input che contiene particelle di dimensioni simili.

Per ogni openint, la somma dei valori dei pixel nell'opening viene calcolato. Questa somma è chiamata \textbf{surface area (superficie)} e diminuisce in funzione all'incremento della dimensione di SE.

Questa procedura produce una matrice 1-D di tali numeri, con ogni elemento nella matrice che diventa uguale alla somma dei pixel nell'opening per la dimensione SE corrispondente a quella posizione nella matrice.

Per enfatizzare i cambiamenti tra successivi opening, si calcola la defferenza tra gli elementi adiacenti della matrice 1-D.

Le visualizzare i risultati, la differenza viene tracciata (plotted). I picchi nel grafico indicano la dimensione predominante della distribuzione di particelle nell'immagine.

\subsection{Segmentazione Texturale}
L'immagine rumorosa qui rappresentata ha due regioni strutturali:
\begin{itemize}
	\item una regione composta da chiazza larghe sulla destra
	\item una regione composta da chiazze piccole sulla sinistra
\end{itemize}

\subsection{Ricostruzione morfologica su scala di grigi}
Sia $f$ il marker e $g$ la maschera dell'immagine. Entrambi sono immagini in tono di grigio della stessa dimensione e $f \leq g$

La \textbf{dilatazione geodetica di dimensione 1} di $f$ rispetto a $g$ è definita come:
$$
D_{g}^{(n)}(f) = D_{g}^{(1)}[D_{g}^{(n-1)}(f)]
$$

dove:
$$
D_{g}^{(0)}(f) = f
$$

L'\textbf{erosione geodetica di dimensione 1} di $f$ rispetto a $g$ è definita come:
$$
E_g^{(1)}(f) = (f \ominus b) \vee g
$$

dove $\vee$ indica " the point-wise maximum operator"

L'\textbf{erosione geodetica di dimensione n} di $f$ rispetto a $g$ è definita come:
$$
E_g^{(n)}(f) = E_g^{(1)}[E_g^{(n-1)}(f)]
$$

dove:
$$
E_g^{(0)}(f) = f
$$

La ricostruzione morfologica su scala di grigi tramite la dilatazione di maschera $g$ di una immagine $f$ è definita come la dilatazione geodetica di $f$ rispetto a $g$, iterata sino a che viene raggiunta la stabilità:
$$
R_g^{(D)}(f) = R_g^{(k)}(f)
$$

con $k$ tale che $D_g^{(k)}(f) = D_g^{(k+1)}(f)$

La ricostruzione morfologica su scala di grigi tramite l'erosione di maschera $g$ di una immagine $f$ è definita come l'erosione geodetica di $f$ rispetto a $g$, iterata sino a che viene raggiunta la stabilità:
$$
R_g^{(E)}(f) = E_g^{(k)}(f)
$$

con $k$ tale che $E_g^{(k)}(f) = E_g^{(k+1)}(f)$

L'opening tramite ricostruzione di dimensione $n$ di una immagine $f$ è definita come la ricostruzione tramite dilatazione di $f$ partendo dall'erosione di dimensione $n$ di $f$:
$$
O_R^{n}(f) = R_f^{(D)}[()f \ominus nb)]
$$

dove $(f \ominus nb)$ denota $n$ erosioni di $f$ tramite $b$.

Il closing tramite ricostruzione di dimensione $n$ di una immagine $f$ è definita come la ricostruzione tramite erosione di $f$ partente dalla dilatazione di dimensione $n$ di $f$:
$$
C_R^{(n)}(f) = R_f^{(E)}[(f \oplus nb)]
$$

dove $f \oplus nb)$ denota $n$ dilatazioni di $f$ tramite $b$.