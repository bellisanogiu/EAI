\chapter{Morfologia}

\section{Introduzione}
La morfologia è una branca della biologia che tratta le forme e la struttura di animali e piante. Nel contesto dell'elaborazione delle immagini, la \textbf{morfologia matematica} è uno strumento che è utile per:

\begin{itemize}
	\item estrarre componenti dall'immagine che sia utili per rappresentare e descrivere regiorni come confini, scheletri, e gusci convessi
	
	\item attuare tecniche di processing come il filtraggio morfologico il thinning e il pruning (potatura)
\end{itemize}

\subsection{Campi di utilizzo della morfologia matematica}
\begin{itemize}
	\item image enhancement
	\item image restoration
	\item noise reduction
	\item space-time filtering
	\item image segmentation
	\item edge detection
	\item texture analysis
	\item particle analysis
	\item component analysis
	\item shape analysis
	\item feature generation
	\item feature detection
	\item skeletonization
	\item general thinning
	\item curve filling
	\item image compression
\end{itemize}

\subsection{Il linguaggio della morfologia matematica}
Il linguaggio alla base della morfologia matematica (denominato MM) è basato sulla \textbf{teoria degli insiemi}.

Si campionano le partizioni $xy$ del piano in una griglia, con le coordinate del centro di ogni griglia che diventano un paio di elementi che formano lo spazio degli interni 2D $ZxZ(x^2)$ tale che è l'insieme di tutte le coppie di elementi ordinati $(a, b), a,b \in Z$

Gli \textbf{insiemi} nella MM rappresentano le forme degli oggetti in una immagine.
\begin{itemize}
	\item l'insieme di tutti i pixel neri in una immagine binaria è una descrizione completa dell'immagine
	
	\item in una immagine binari gli insiemi in questione sono membri $Z^2$ dove ogni elemento di un insieme è una 2-tupla (o un 2-D vettore) le cui le coordinate $x, y$ sono le coordinate di un pixel bianco (o nero a seconda della convenzione adottata) nell'immagine
\end{itemize}

\subsection{Altri concetti della teoria degli insiemi}

\begin{itemize}
	\item sia $A$ un insieme in $z^2$ con elementi $(x, y)$:
	\begin{itemize}
		\item se $w = (x, y)$ è un elemento di $A, \quad w \in A$
		\item se $w = (x, y)$ non è un elemento di $A, \quad w \notin A$
	\end{itemize}
	
	\item un insieme di pixel $B$ che soddisfa una particolare condizione è scritto come:
	$$
	B = \{w | \text{condizione}\}
	$$
	
	\item il \textbf{complemento di A} è l'insieme di tutte le coordinate di pixel che non appartengono ad $A$ è indicato con $A^c$:
	$$
	A^C = \{w | w \notin A \}
	$$
	
	\item l'\textbf{unione} di due insiemi $A$ e $B$ è l'insieme formato da tutti gli elementi di $A$ e $B$ ed è indicato come:
	$$
	C = A \cup B
	$$
	
	\item l'\textbf{intersezione} di due insiemi $A$ e $B$ è l'insieme formato da tutti gli elementi che appartengono sia ad $A$ che a $B$ ed è indicato come:
	$$
	C = A \cap B
	$$
	
	\item la \textbf{differenza} di due insiemi $A$ e $B$ è l'insieme degli elementi che appartengono ad $A$ ma non a $B$ ed è indicato come:
	$$
	A – B = \{w | w \in A, w \notin B\}
	$$
	
	\item la \textbf{riflessione} di un insieme $B$ indicato con $\hat B$:
	$$
	\hat B = \{w | w = -b, \quad b \in B\}
	$$
	
	\item la \textbf{traslazione} di un insieme $B$ di un punto $z =(z_1, z_2)$ è indicato con $(B)_z$ e definito come:
	$$
	(B)_z = \{c|c= b + z, \quad b \in B \}
	$$
\end{itemize}

\subsection{Altri concetti dalla teoria degli insiemi}
\begin{itemize}
	\item una immagine binaria può essere vista come una funzione bivalued di $x$ e $y$
	
	\item per la teoria MM una immagine binaria è un insieme di pixel in primo piano (1-valued), che si trovano in $Z^2$
	
	\item le operazioni come unione e intersezione possono essere applicate direttamente gli insiemi di immagini binarie
	
	\item le operazioni di riflessione e traslazione sono impiegate spesso nella morfologia per formulare operazioni basate sugli \textbf{elementi strutturati} (SEs)
	\begin{itemize}
		\item SEs sono piccoli insiemi di sottoimmagini usate per sondare un'immagine e studiarne le proprietà desiderate
		
		\item SEs quando lavorano con le immagini, sono degli array rettangolari e ciò si ottiene aggiungendo il minor numero possibile di elementi di sfondo necessari per formare una matrice rettangolare
		
		\item si crea un nuovo insieme facendo scorrere $B$ su $A$ in modo che l'origine di $B$ visiti ogni elemento di $A$
		
		\item ogni parte dell'origine $B$, se questo è contenuto in $BA$, marca la parte come membro di in un nuovo insieme (indicato come ombreggiato), altrimenti non viene marcato come parte di un nuovo elemento (indicato come non ombreggiato)
		
		\item il risultato è che il confine dell'insieme viene \textbf{eroso}
	\end{itemize}
\end{itemize}

