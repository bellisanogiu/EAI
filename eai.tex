\documentclass[a4paper,10pt,oneside]{book}
\usepackage[utf8]{inputenc}
\usepackage[italian]{babel}
\usepackage[T1]{fontenc}
\usepackage{lmodern}
\usepackage{hyperref}

\usepackage{microtype}				% Microgiustificazione
\usepackage{makeidx}				% indice analitico
\usepackage{hyperref}				% Gestione riferimenti incrociati

\title{Elaborazione e Analisi delle Immagini}
\author{Giuseppe Bellisano}
\date{30 novembre 2016}

\makeindex

\usepackage[nonumberlist]{glossaries} % glossario
%\glossarystyle{altlisthypergroup}
	\makeglossaries
	
\newglossaryentry{trasformazione geometrica affine}
{
	name=trasformazione geometrica affine, description={Esempi di affinità sono rotazioni, omotetie, traslazioni, rototraslazioni, riflessioni. Le affinità non sono necessariamente isometrie, non preservano cioè angoli e distanze, mentre mantengono sempre il parallelismo tra le rette.}
}
% --------------------------------------------------------------------
\newglossaryentry{omega}{%
	name=$\omega_\mathrm{n}$,
	description={pulsazione naturale},
}
% --------------------------------------------------------------------
\newglossaryentry{omega1}{%
	name=$\omega_\mathrm{n}$,
	description={pulsazione naturale},
}
% --------------------------------------------------------------------
\newglossaryentry{omega2}{%
	name=$\omega_\mathrm{n}$,
	description={pulsazione naturale},
}



\usepackage{listings}
\usepackage{color}
\definecolor{mygreen}{rgb}{0,0.6,0}
\definecolor{mygray}{rgb}{0.5,0.5,0.5}
\definecolor{mymauve}{rgb}{0.58,0,0.82}

\lstset{ %
  backgroundcolor=\color{white},   % choose the background color; you must add \usepackage{color} or \usepackage{xcolor}; should come as last argument
  basicstyle=\footnotesize,        % the size of the fonts that are used for the code
  breakatwhitespace=false,         % sets if automatic breaks should only happen at whitespace
  breaklines=true,                 % sets automatic line breaking
  captionpos=b,                    % sets the caption-position to bottom
  commentstyle=\color{mygreen},    % comment style
  deletekeywords={...},            % if you want to delete keywords from the given language
  escapeinside={\%*}{*)},          % if you want to add LaTeX within your code
  extendedchars=true,              % lets you use non-ASCII characters; for 8-bits encodings only, does not work with UTF-8
  frame=single,	                   % adds a frame around the code
  keepspaces=true,                 % keeps spaces in text, useful for keeping indentation of code (possibly needs columns=flexible)
  keywordstyle=\color{blue},       % keyword style
  language=Matlab,                 % the language of the code
  morekeywords={*,...},           % if you want to add more keywords to the set
  numbers=left,                    % where to put the line-numbers; possible values are (none, left, right)
  numbersep=5pt,                   % how far the line-numbers are from the code
  numberstyle=\tiny\color{mygray}, % the style that is used for the line-numbers
  rulecolor=\color{black},         % if not set, the frame-color may be changed on line-breaks within not-black text (e.g. comments (green here))
  showspaces=false,                % show spaces everywhere adding particular underscores; it overrides 'showstringspaces'
  showstringspaces=false,          % underline spaces within strings only
  showtabs=false,                  % show tabs within strings adding particular underscores
  stepnumber=2,                    % the step between two line-numbers. If it's 1, each line will be numbered
  stringstyle=\color{mymauve},     % string literal style
  tabsize=2,	                   % sets default tabsize to 2 spaces
  title=\lstname                   % show the filename of files included with \lstinputlisting; also try caption instead of title
}

\begin{document}
\maketitle
\tableofcontents 					%indice generale

\section{Image segmentation} \label{sec:seg} \index{Image segmentation} 
\subsection{Cosa è l'image segmentation?}
La segmentazione è uno dei passi più importanti nell'analisi delle immagini processate

Il suo scopo è quello di \textbf{dividere} l'immagine in parti che hanno una forte correlazione con oggetti o aree che si trovano nel mondo reale.

\subsection{Quanti tipi di segmentazione abbiamo?}
Abbiamo due tipi di segmentazione:

\begin{enumerate}
\item \textbf{Segmentazione Completa} in cui le regioni corrispondono direttamente con gli oggetti dell'immagine in ingresso. In pratica ogni elemento dell'immagine viene riconosciuto e isolato. Per ottenere questo risultato è necessario avere informazioni di processing di alto livello (high level processing) da utilizzare per comprendere il dominio del problema da risolvere (che ci permette di riconoscere gli oggetti/regioni)

\item \textbf{Segmentazione Parziale} in cui le regioni/segmenti di immagini non corrispondono direttamente con gli oggetti reali rappresentati nell'immagine. In questo caso l'immagine è divisa in regioni omogenee secondo una determinata proprietà scelta (colore, texture, riflettività).

La segmentazione parziale è seguita da un ulteriori processi sino ad arrivare all'immagine finale di segmentazione che può essere trovata grazie all'aiuto di informazioni di alto livello.
\end{enumerate}

\subsection{In quali campi viene utilizzata la segmentazione?}
Ecco alcune delle problematiche più comuni che possono essere risolte con la segmentazione:
\begin{itemize}
\item ottenere dati da immagini ambigue
\item rumore nelle informazioni
\item ottenere oggetti con discreto contrasto su di uno sfondo uniforme
\item semplici lavori di raggruppamento come cellule del sangue, caratteri stampati
\end{itemize}

In generale è difficile ottenere una corretta e completa segmentazione; per questo spesso di utilizza una segmentazione parziale come input in un processo di alto livello.

\subsection{Quali sono i metodi utilizzati per la segmentazione?}
\begin{itemize}
\item Thresholding (Approccio Globale) che si basa sull'istogramma di alcune caratteristiche dell'immagine come il \textbf{gray level
thresholding} o \textbf{soglia del livello di grigio}. Si ricercano i toni soglia che separano oggetti differenti nell’immagine. E’ il metodo più semplice e veloce.
\item Segmentazione basata sugli Edge in cui ci cercano i contorni degli oggetti basandosi sulla discontinuità dei toni, cioè sugli edge
\item Segmentazione basata sulle Regioni in cui si ricercano le regioni, cioè degli oggetti uniformi secondo un certo criterio (stesso tono di grigio, o tono di grigio che si differenzia al massimo di una certa soglia)
\end{itemize}

\subsection{Cosa si intende con Approccio Globale: Gray level thresholding?}
E' il sistema più semplice a livello concettuale di segmentazione e per questo è anche il sistema più efficiente computazionalmente e veloce.

Esso viene applicato alle immagini che contengono regioni o oggetti con un costante livello di riflettività o assorbimento di luce. Si sfrutta questa caratteristica per partizionare l'immagine grazie ai valori di intensità e/o alle proprietà di questi valori.

\subsection{Quale è l'idea alla base del gray level thresholding con due elementi (mode) da separare?}

\begin{enumerate}
\item Si supponga di avere un istogramma di intensità di una immagine \textit{f(x,y)} che è composta da un oggetto bianco (level 255) e di uno sfondo scuro (level 0) e che si voglia estrarre l'oggetto dallo sfondo.

\item Inizialmente si seleziona una soglia \textit{T} con cui si ottiene l'immagine segmentata \textit{g(x, y)}:

$$
g(x, y) =
\bigg \{
\begin{array}{rl}
1 & se \quad f(x, y) > T \\
0 & se \quad f(x, y) \leq T \\
\end{array}
$$

\item Se il valore di intensità dell'immagine è superiore a quello della soglia, si assegna a \textit{g(x, y) il valore 1, altrimenti 0}
\end{enumerate}

\subsection{Quale è l'idea alla base del gray level thresholding con tre elementi (mode) da separare?}

\begin{itemize}
\item Maggiore è il numero di elementi (mode), più complessa sarà la segmentazione. Esaminiamo il caso di una immagine con due oggetti chiari posti su uno sfondo scuro.

\item L'immagine segmentata risultante \textit{g(x, y)} sarà:

$$
g(x, y) =
\bigg \{
\begin{array}{rl}
a & se \quad f(x, y) > T_2 \\
b & se \quad  T_1 < f(x, y) \leq T_2 \\
c & se \quad f(x, y) \leq T_1 \\
\end{array}
$$

dove \textit{a, b, c} sono tre diversi valori di intensità e $T_1$ e $T_2$ sono due differenti valori di soglia.

\end{itemize}


\subsection{Nel gray level thresholding quanti e quali tipi di soglia possiamo avere?}
Possiamo avere due tipi di soglia:

\begin{enumerate}
\item Soglia Globale
\item Soglia Variabile, che si divide in soglia locale o regionale
\end{enumerate}

\subsection{Il thresholding applicato ad immagini con rumore e variazioni di luminosità: cosa cambia?}
Esaminando l'istogramma dell'immagine si nota che questo subisce delle variazioni a seconda che 
l'immagine si affetta da rumore o una intensità variabile.

\subsection{Cosa è il single global threshold e quando si applica?}
Il single global threshold si applica a quelle immagini in cui l'intensità degli oggetti e dello sfondo rende questi facilmente distinti. L'algoritmo che si applica all'intera immagine si basa sui seguenti passi:

\begin{enumerate}
\item Si seleziona una soglia \textit{T}
\item Si effettua la segmentazione dell'immagine usando la soglia \textit{T} producendo due gruppi di pixel:

\begin{itemize}
\item $G_1$ dato da tutti i pixel che hanno una intensità $\geq T$
\item $G_2$ dato da tutti i pixel che hanno una intensità $ < T$
\end{itemize}

\item Si calcolano i valori di \textbf{intensità medi} $m_1$ e $m_2$ rispettivamente per i gruppi di pixel $G_1$ e $G_2$
\item Si determina un nuovo livello di soglia \textit{T} secondo la formula:

$$
T = {1 \over 2} (m_1 + m_2)
$$

\item Si ripetono i passi dal 2 al 4 sino a quando la differenza tra i valori di \textit{T} (delle successive iterazioni) è più piccolo rispetto al valore predefinito $T_0$
\end{enumerate}

\subsection{Quali sono le caratteristiche del single global threshold?}
\begin{itemize}
\item E' un algoritmo semplice.
\item Funziona bene nei casi in cui l'istogramma dell'immagine presenta delle campane (le mode) corrispondenti agli oggetti e allo sfondo separate da una valle.
\item Il valore di soglia iniziale deve essere maggiore rispetto al minimo valore di intensità e minore rispetto al massimo valore di intensità nell'immagine.
\end{itemize}

\subsection{Qual è l'algoritmo Matlab per il global thresholding?}
Indicando con:

\begin{itemize}
\item \textit{f} per l'immagine di input
\item \textit{g} per l'immagine di output
\item \textit{T} per la soglia
\end{itemize}

\begin{lstlisting}
function [g]=iter_thresh(f)
T = 0.5 * (double(min(f(:)))+ double(max(f(:))));
flag = false;
while ~flag
	g = f >= T;
	Tnext = 0.5 * (mean(f(g)) + mean(f(~g)));
	flag = abs (T - Tnext) < 0.5;
	T = Tnext;
end
\end{lstlisting}

\section{Cosa è l'Optimum global thresholding o metodo Otsu?} \index{Otsu}
Meglio conosciuto come \href{https://en.wikipedia.org/wiki/Otsu\%27s_method}{Metodo Otsu} è un è un metodo di sogliatura automatica dell'istogramma nelle immagini digitali.

Viene definito come "ottimo" perché \textbf{massimizza la varianza tra le classi}. Un algoritmo di sogliatura (threshold) che fornisce la migliore separazione tra le classi in termini dei loro valori di intensità è il miglior threshold.

\subsection{Come funziona il metodo Otsu?} \index{Otsu}

\begin{itemize}
\item Si considerino: 

\begin{itemize}
\item \{0, 1, 2, ... , L-1\} siano gli \textit{L} distinti livelli di intensità in una immagine
\item MxN sia la dimensione dell'immagine di \textit{M} righe X \textit{N} colonne
\item $n_i$ indica il numero di pixel con intensità \textit{i}
\item $MN = n_0 + n_1 + n_2 + \dot{}, + n_{L-1}$
\end{itemize}

\item l'istogramma normalizzato ha componenti $p_i = {n_i \over MN}$ da cui segue:
$$\sum_{i=0}^{L-1} p_i = 1 \quad p_i \geq 0$$

\item Selezionando e utilizzando una soglia $T(k) = k$ con $0 < k < L-1$ con l-immagine di input, si ottengono le due classi $c_1$ e $C_2$ dove:

\begin{itemize}
\item $C_1$ è la classe di tutti i pixel con valori di intensità compresi nel range $[0, k]$
\item $C_2$ è la classe di tutti i pixel con valori di intensità compresi nel range $[k+1, L-1]$
\end{itemize}

\item Usando questa soglia, si può calcolare la \textit{probabilità} $P_1(k)$ che un pixel appartenga alla classe $C_1$:

$$
P_1(k) = \sum_{i=0}^{k} p_i
$$

\item La \textit{probabilità} che il pixel invece appartenga alla classe $C_2$ è data da:

$$
P_2(k) = \sum_{i=k+1}^{L-1} p_i = 1 - P_1(k)
$$

\item Il valore di \textit{intensità medio} di tutti i pixel della classe $C_1$ è:

$$
m_1(k) = \sum_{i=0}^{k} i P({i \over C_1}) = \sum_{i=0}^{k} i P({C_1 \over i}) {P(i) \over P(C_1)} = {1 \over P_1(k)} \sum_{i=0}^{k} i p_i
$$
\item Il valore di \textit{intensità medio} di tutti i pixel della classe $C_2$ è invece:

$$
m_2(k) = \sum_{i=k+1}^{L-1} i P({i \over C_2}) = {1 \over P_2(k)} \sum_{i=k+1}^{L-1} i p_i
$$


\item La \textit{media complessiva} sino al livello \textit{k} è data da:

$$
m(k) = \sum_{i=0}^{k} i p_i
$$

\item Invece l'\textit{intensità media} dell'intera immagine (\textbf{la media globale}) è data da:

$$
m_G = \sum_{i=0}^{L-1} i p_i
$$ 

\item Inoltre valgono le seguenti:

$$
P_1m_1 + P_2m_2 = m_G
$$

$$
P_1 + P_2 = 1
$$

\item Per valutare la bontà della soglia al livello \textit{k} si usa la \textit{metrica normalizzata}: 

$$
\eta = {\sigma_{B}^{2} \over \sigma_{G}^{2}}
$$

\begin{itemize}
\item dove $\sigma_{G}^{2}$ è la \textbf{varianza globale} ed è \textit{costante}:

$$
\sigma_{G}^{2} = \sum_{i=0}^{L-1} (i - m_G)^2 p_i
$$

\item dove $\sigma_{B}^{2}$ è la \textbf{varianza tra le classi} che indica la misura della \textit{separabilità} tra le classi:

$$
\sigma_{B}^{2} = P_1(m_1 - m_G)^2 + P_2(m_2 - m_G)^2 = P_1 P_2(m_1 - m_2)^2 = {(m_G P_1 - m)^2 \over P_1 (1 - P_1)}
$$
\end{itemize}

\item (The farther the two means m 1 and m 2 are from each other the larger will be.)

\item Poiché $\sigma_{G}^{2}$ è una costante ne consegue che $\eta$ è una \textit{misura della separabilità}. Inoltre massimizzare questa metrica, è identico a massimizza $\sigma_{B}^{2}$

\item l'obiettivo principale è determinare il valore \textit{k} di soglia che massimizza la varianza di classe.

\item Introducendo nuovamente \textit{k} si ha:

$$
\eta(k) = {\sigma_{B}^{2} \over \sigma_{G}^{2}}
$$

e

$$
{\sigma_{B}^{2}(k)} = {[{m_g P_1(k} - m(k)]^2 \over {P_1(k}[1 - P_1(k)]}
$$

\item La \textbf{soglia ottimale} è il valore $k^*$ che massimizza $\sigma_{B}^{2}(k)$:

$$
\sigma_{B}^{2}(k^*) = \max_{0 \leq k \leq L-1} \sigma_{B}^{2}(k)
$$

\item Per trovare $K^*$ si deve valutare questa equazione per tutti i valori interi di \textit{k} e scegliere il valore che produce il massimo  $\sigma_{B}^{2}(k)$

\item Se il massimo esiste \textit{per più di un valore di k} è consuetudine mediare i diversi valori di \textit{k} per cui $\sigma_{B}^{2}(k)$ è massima.

\item La metrica normalizzata$\eta$ valutata per il valore di soglia ottimale $\eta(k^*)$ può essere utilizzata per ottenere una stima quantitativa della separabilità delle classi, che a sua volta fornisce un'idea della facilità di
sogliatura di una data immagine. Il suo valore oscilla tra $0$ e $1$:

\begin{itemize}
\item il valore più basso è ottenibile solo da immagini con un singolo livello di intensità costante
\item Il limite superiore invece è ottenibile solo dalle immagini 2-valori con intensità uguali a $0$ e $L-1$
\end{itemize}

\end{itemize}

\subsection{Spiegare le sogliature multiple}
Il metodo di Otsu può essere esteso ad un numero arbitrario di soglie in modo da valutare un diverso numero di classi.

Generalmente i metodi che utilizzano più di due soglie vengono risolte con più valori di intensità (colore).

Un altro metodo è quello di suddividere l'immagine in \textit{rettangoli sovrapposti}

\begin{itemize}
\item questo sistema viene utilizzato per compensare i casi di illuminazione non uniforme
\item i rettangoli vengono scelti di piccole dimensioni in modo che l'illuminazione da essi coperta, sia uniforme
\end{itemize}

\subsection{Illustra il metodo Otsu in Matlab} \index{Otsu}
La funzione \textit{graythresh} di Matlab permette di calcolare una soglia utilizzando il metodo di Otsu.

\subsection{Tecniche per migliorare il global thresholding}
\begin{itemize}
\item Applicare lo smoothing all'immagine prima del thresholding: è una tecnica che si applica quando il rumore non può essere ridotto e come metodo di sogliatura si seleziona il global thresholding. Di seguito un esempio di codice Matlab.

\begin{lstlisting}
f=imread('large_septagon_gaussian_noise_mean_0_std_50_added.tif');
figure, imshow(f);
figure, imhist(f);
T=graythresh(f);
T=T*255
g=f>=T;
figure, imshow(g);
filt=fspecial('average', [5 5]);
ff=imfilter(f, filt);
figure, imshow(ff);
figure, imhist(ff);
ylim([0 16000])
T1=graythresh(ff);
T1=T1*255
g1=ff>=T1;
figure, imshow(g1);
\end{lstlisting}

\item Thresholding variabile attraverso il partizionamento dell'immagine. Di seguito un esempio di codice Matlab.

\begin{lstlisting}


\end{lstlisting}

\end{itemize}

\chapter{Image processing con Matlab}

\section{Ridimensionamento}

\textbf{comando:} imresize

\textbf{possibili domande:}
\begin{itemize}
	\item Leggere un'immagine e rimpicciolirla o farne lo zoom di un valore intero
	\item Leggere un'immagine e rimpicciolirla o farne lo zoom di un valore decimale con interpolazione lineare
\end{itemize}

\section{Trasformazioni geometriche}
In Matlab è possibile effettuare una \gls{trasformazione geometrica affine}
specificando la matrice di trasformazione \textit{T} attraverso il comando \textit{maketform}.
Per effettuare la trasformazione si usa il comando \textit{imtransform}.

\textbf{comando:} maketform, imtransform, imrotate, impixelinfo (pixval),

\textbf{possibili domande:}
\begin{itemize}
	\item Effettuare la rotazione di un’immagine qualsiasi
	\item Confrontare il risultato con quello ottenuto mediante la funzione \textit{imrotate}
	\item Leggere l'immagine di 'lena' e realizzare l'ingrandimento di una zona dell’immagine usando la matrice di trasformazione \textit{T}
	la sezione da ingrandire è intorno all'occhio di lena.
	Per individuare la sezione e, quindi, avere informazioni sulla posizione dei pixel potete usare il comando \textit{impixelinfo} o \textit{pixval} (in base alla versione di Matlab più o meno recente).
	\item Fare degli esperimenti modificando il tipo di interpolazione e notate l'effetto di blocchettatura causa
	
	\item La combinazione di diverse trasformazioni affini è ancora una trasformazione 	affine, che può essere ottenuta tramite il prodotto (matriciale) delle matrici che le definiscono.
	
	\item Scrivere una funzione dal prototipo \verb|function g=rot_dist(f, alfa, c)| per realizzare prima una rotazione e poi una distorsione verticale.
	
	\item Creare l'immagine di ingresso usando il seguente comando
	\verb|f = checkerboard(40);| in modo da generare una scacchiera su cui le modifiche risultano essere più facilmente visibili.
\end{itemize}

\section{Rumore}
\begin{itemize}
	\item Aggiungere del rumore gaussiano bianco ad un immagine f con il comando	\verb|noisy = f + n con n = d*randn(size(f))| dove \textbf{d è la deviazione standard del rumore}.
	
	\item Rimuovere il rumore dall'immagine con i filtri a media mobile (al variare della dimensione della finestra).
	
	\item Valutare l'efficacia del filtraggio sia visivamente sia calcolando l'errore quadratico medio tra f e l'immagine "ripulita" .
	
	\item L’errore quadratico medio rappresenta una misura quantitativa per stabilire quanto l'immagine elaborata sia simile all'originale.
	
	\item L'MSE (Mean Squared Error) tra due immagini si definisce come:
	$$
	MSE = {1 \over MN} \sum_{m=0}^{M-1} \sum_{n=0}^{N-1} |f(m, n) - g(m, n)|^2
	$$
\end{itemize}

\section{Smoothing con thresholding}

\begin{itemize}
	\item Consideriamo l'immagine 'telescopio.jpg', proveniente dal telescopio Hubble, in orbita intorno alla terra. Rilevare gli oggetti grandi realizzando le seguenti operazioni:
	\begin{itemize}
		\item Visualizzare l'immagine
		\item Applicare il filtro che effettua la media aritmetica su una finestra di dimensioni 15x15 e visualizzare il risultato
		\item Applicare un'operazione a soglia per eliminare gli oggetti piccoli (considerare una soglia pari al 25 per cento del valore massimo presente nell'immagine filtrata)
		\item Visualizzare il risultato dell'elaborazione
		
	\end{itemize}
\end{itemize}

\printindex				% stampa indice analitico

\printglossaries		% stampa glossario
\end{document}


