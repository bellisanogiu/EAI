\chapter{Image processing con Matlab}

\section{Ridimensionamento}

\textbf{comando:} imresize

\textbf{possibili domande:}
\begin{itemize}
	\item Leggere un'immagine e rimpicciolirla o farne lo zoom di un valore intero
	\item Leggere un'immagine e rimpicciolirla o farne lo zoom di un valore decimale con interpolazione lineare
\end{itemize}

\section{Trasformazioni geometriche}
In Matlab è possibile effettuare una \gls{trasformazione geometrica affine}
specificando la matrice di trasformazione \textit{T} attraverso il comando \textit{maketform}.
Per effettuare la trasformazione si usa il comando \textit{imtransform}.

\textbf{comando:} maketform, imtransform, imrotate, impixelinfo (pixval),

\textbf{possibili domande:}
\begin{itemize}
	\item Effettuare la rotazione di un’immagine qualsiasi
	\item Confrontare il risultato con quello ottenuto mediante la funzione \textit{imrotate}
	\item Leggere l'immagine di 'lena' e realizzare l'ingrandimento di una zona dell’immagine usando la matrice di trasformazione \textit{T}
	la sezione da ingrandire è intorno all'occhio di lena.
	Per individuare la sezione e, quindi, avere informazioni sulla posizione dei pixel potete usare il comando \textit{impixelinfo} o \textit{pixval} (in base alla versione di Matlab più o meno recente).
	\item Fare degli esperimenti modificando il tipo di interpolazione e notate l'effetto di blocchettatura causa
	
	\item La combinazione di diverse trasformazioni affini è ancora una trasformazione 	affine, che può essere ottenuta tramite il prodotto (matriciale) delle matrici che le definiscono.
	
	\item Scrivere una funzione dal prototipo \verb|function g=rot_dist(f, alfa, c)| per realizzare prima una rotazione e poi una distorsione verticale.
	
	\item Creare l'immagine di ingresso usando il seguente comando
	\verb|f = checkerboard(40);| in modo da generare una scacchiera su cui le modifiche risultano essere più facilmente visibili.
\end{itemize}

\section{Rumore}
\begin{itemize}
	\item Aggiungere del rumore gaussiano bianco ad un immagine f con il comando	\verb|noisy = f + n con n = d*randn(size(f))| dove \textbf{d è la deviazione standard del rumore}.
	
	\item Rimuovere il rumore dall'immagine con i filtri a media mobile (al variare della dimensione della finestra).
	
	\item Valutare l'efficacia del filtraggio sia visivamente sia calcolando l'errore quadratico medio tra f e l'immagine "ripulita" .
	
	\item L’errore quadratico medio rappresenta una misura quantitativa per stabilire quanto l'immagine elaborata sia simile all'originale.
	
	\item L'MSE (Mean Squared Error) tra due immagini si definisce come:
	$$
	MSE = {1 \over MN} \sum_{m=0}^{M-1} \sum_{n=0}^{N-1} |f(m, n) - g(m, n)|^2
	$$
\end{itemize}

\section{Smoothing con thresholding}

\begin{itemize}
	\item Consideriamo l'immagine 'telescopio.jpg', proveniente dal telescopio Hubble, in orbita intorno alla terra. Rilevare gli oggetti grandi realizzando le seguenti operazioni:
	\begin{itemize}
		\item Visualizzare l'immagine
		\item Applicare il filtro che effettua la media aritmetica su una finestra di dimensioni 15x15 e visualizzare il risultato
		\item Applicare un'operazione a soglia per eliminare gli oggetti piccoli (considerare una soglia pari al 25 per cento del valore massimo presente nell'immagine filtrata)
		\item Visualizzare il risultato dell'elaborazione
		
	\end{itemize}
\end{itemize}