\chapter{Segmentazione Watershed in Matlab}
La trasformata watersher è calcolata dalla funzione $watershed$, la cui sintassi è la seguente:
$$
L = watershed(f)
$$

dove $L$ è una etichetta di matrice con interi positivi che corrispondo ai bacini e con i valori zero usati per indicare le creste watershed.

Uno strumento comunemente utilizzato con la trasformata watershed per la segmentazione è la \textbf{trasformata della distanza}.

La trasformate della distanza di una immagine binaria è la distanza da ogni pixel dal pixel più vicino di valore diverso da zero.

La trasformata della distanza può essere calcolata usando la funzione $bwdist$, la cui sintassi è:
$$
D = bwdist(f)
$$

\section{Segmentazione di una immagine binaria}
\begin{lstlisting}
f = imread('binary-dowel-image.tif');
figure, imshow(f), title('Binary image');

fc = ~f ;
figure, imshow(fc), title('Complement image');

D = bwdist(fc);
figure, imshow(imadjust(uint16(D))), title('Distance transform');

L = watershed(-D);
w = L==0;
figure, imshow(w)
title('Watershed lines of the negative of the distance transform');

g2 = f & ~w ;
figure, imshow(g2)
title('Watershed lines superimposed in black over original binary image');
\end{lstlisting}

\section{Segmentazione di una immagine in toni di grigio}
\begin{lstlisting}
f = imread('small-blobs.tif');
figure, imshow(f), title('Original image');

h = fspecial('sobel');
fd = double(f);
g = sqrt(imfilter(fd, h, 'replicate').^2 + imfilter(fd, h, 'replicate').^2);
figure, imshow(imadjust(uint16(g))), title('Gradient image');

L = watershed(g);
wr = L==0;
figure, imshow(wr), title('Watershed segmentation');

g2 = imclose(imopen(g, ones(3,3)), ones(3,3));
L2 = watershed(g2);
wr2 = L2==0;
f2=f;
f2(wr2)=255;
figure, imshow(f2)
title('Watershed segmentation on smoothed image');
\end{lstlisting}

\section{Segmentazione usando i marcatori}
Per controllare la sovra-segmentazione si utilizzano i marker (marcatori) interni ed esterni.

La funzione IPT $imregionalmin$ calcola la posizione di tutti i minimi locali in una immagine. La sintassi è la seguente:
$$
rm = imregionalmin(f)
$$

dove $f$ è l'immagine in toni di grigio e $rm$ è una immagine binaria in cui i pixel in primo piano marcano le posizioni del minimo locale; questi ultimi possono rappresentare dettagli irrilevanti.

Per eliminare i minimi estranei si utilizza la funzione IPT $imextendedmin$ che calcola l'insieme di punti "bassi" nell'immagine that are deeper than their immediatesurroundings (by a certain height threshold). 

Fornendo sia i marcatori interni che esterni, è possibile modificare l'immagine del gradiente usando una procedura chiamata \textbf{minima imposition}. Questa tecnica modifica una immagine in toni di grigio in modo che i minimi locali si verificano solo nelle posizioni marcate. Altri calori dei pixel sono spinti verso l'alto per rimuovere tutti gli altri minimi locali

