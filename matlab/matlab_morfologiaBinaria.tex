\chapter{Morfologia matematica binaria}

\section{Funzioni}

\subsection{bwmorph}
Questa funzione implementa diverse operazioni basate sulla combinazione di erosione e dilatazione. La sintassi è:
\begin{lstlisting}
g = bwmorph(f, operation, n)
\end{lstlisting}

\begin{itemize}
	\item $f$ è un'immagina binaria di input
	\item $operation$ è una stringa che specifica le operazioni desiderate
	\item $n$ (opzionale) è un numero intero positivo che indica il numero di volte che l'operazione deve essere ripetuta. Se $n$ viene omesso, l'operazione viene eseguita una sola volta
\end{itemize}

\subsection{imreconstruct}
Implementa la ricostruzione morfologica ed è esplicitata dalla sintassi:
\begin{lstlisting}
out = imreconstruct(marker, mask)
\end{lstlisting}

\begin{itemize}
	\item $marker$ è l'immagine marcatore
	\item $mask$ è l'immagine maschera
\end{itemize}

\subsection{imclearborder}
Permette la pulizia dei bordi degli oggetti, con la sintassi:
\begin{lstlisting}
g = imclearborder(f, conn)
\end{lstlisting}

\begin{itemize}
	\item $f$ è l'immagine di input
	\item $g$ è il risultato
	\item $conn$ può assumere i valori di 4 oppure 8 (valore di default)
\end{itemize}

\section{Esercizi}

\subsection{Esercizio 1}
Calcola e visualizza il centro di massa di ogni componente connesso.
\begin{lstlisting}
f= imread('ten-objects.tif');
imshow(f);
[L, n]=bwlabel(f);
n
hold on; % to plot on the top of the image

for k=1:n
	[r,c] = find(L==k);
	rbar=mean(r);
	cbar=mean(c);
	plot(cbar, rbar, 'Marker', 'o', 'MarkerEdgeColor', 'k', 'MarkerFaceColor', 'k', 'Markersize', 10);
	plot(cbar, rbar, 'Marker', '*', 'MarkerEdgeColor', 'w');
end
\end{lstlisting}

\subsection{Esercizio 2}
Effettua il thinning dell'immagine $finger-print.tif$ sino ad ottenere la stabilità.
\begin{lstlisting}
f=imread('noisy-fingerprint.tif');
imshow(f);
se = strel('square',3);
fo = imopen(f,se);
figure, imshow(fo);
foc = imclose(fo,se);
figure, imshow(foc);

g1 =bwmorph(foc,'thin',1);
figure, imshow(g1);
g2 =bwmorph(foc,'thin',2);
figure, imshow(g2);
ginf =bwmorph(foc,'thin',Inf);
figure, imshow(ginf);
\end{lstlisting}

\subsection{Esercizio 3}
Effettuare la sclettrizzazione dell'immagine $bone$ e si rimuovano i punti finali (5 pixel).
\begin{lstlisting}
f=imread('bone.tif');
imshow(f)
fs = bwmorph(f,'skel',Inf);
figure, imshow(fs);
for k=1:5
	fs=bwmorph(fs,'spur');
end
figure, imshow(fs);
\end{lstlisting}

\subsection{Esercizio 4}
Ricostruzione morfologica
\begin{lstlisting}
marker =imread('recon-marker.tif');
mask = imread('recon-mask.tif');
figure, imshow(mask);
figure, imshow(marker);
recon = imreconstruct(marker, mask);
??? MARKER pixels must be <= MASK
pixels.

Error in ==> imreconstruct at 71
im = imreconstructmex(marker,mask);

mm=marker==255;
mm= uint8(mm*255);
recon = imreconstruct(mm, mask);
figure, imshow(recon);
\end{lstlisting}


\subsection{Esercizio 5}
Effettuare la ricostruzione morfologica per trovare quali caratteri contengono un'asticella verticale.
\begin{lstlisting}
f=imread('book-text.tif');
figure, imshow(f);

fe=imerode(f,ones(51,1));
figure, imshow(fe);

fo=imopen(f, ones(51,1));
figure, imshow(fo);

fobr=imreconstruct(fe,f);
figure, imshow(fobr);
\end{lstlisting}


\subsection{Esercizio 6}
Effettuare una pulizia dei caratteri che toccano lo schermo.
\begin{lstlisting}
g = imclearborder(f);
figure, imshow(g);
figure, imshow(f - g);
\end{lstlisting}