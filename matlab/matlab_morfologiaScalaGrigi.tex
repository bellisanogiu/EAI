\chapter{Morfologia matematica scala di grigi}
Nella morfologia scala di grigi si utilizzano funzioni discrete definite in $Z^2$ nella forma:
\begin{itemize}
	\item $f(x, y)$ che indica una immagine in scala di grigi
	
	\item $b(x, y)$ che indica un SE, utilizzata come sonda per esaminare un'immagine e le sue specifiche proprietà
	\begin{itemize}
		\item gli SE nella morfologia scala di grigi può essere $flat$ oppure $nonflat$
		\item l'origine deve essere definita e solitamente risulta essere simmetrica con l'origine situata al centro
		
		\item la riflessione di SE è indicata come:
		$$
		f
		$$
	\end{itemize}
\end{itemize}